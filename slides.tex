% Presentation structure
%
% Part I: Preliminaries
%   Introduction to @Web
%   @Web core ontology
%   Annotated tables
%   Examples of constraints we want to verify automatically
%
% Part II: RDF data validation: survey of the state of the art
%   Shape Expressions
%   SHACL
%   Plain SPARQL
%
% Part III: Implementation
%   Examples of actual constraints
%   Live demo
%
% Part IV: Conclusions
%   Conclusions
%   Future work
%   Thanks


\documentclass{beamer}
\usepackage[utf8]{inputenc}
\usepackage{graphicx}

% Remove navigation controls
\usenavigationsymbolstemplate{}

% Slide numbering
\setbeamertemplate{footline}[frame number]

\title{Using SPARQL queries to express integrity constraints in RDF graphs}
\subtitle{Final internship report}
\author{
  Leandro Lovisolo
}
\date{February 25, 2016}
\institute{
  INRA SupAgro and INRIA GraphiK \\
  Montpellier, France
}

\begin{document}

\begin{frame}
  \titlepage
\end{frame}

% \begin{frame}
%   \frametitle{Motivation}
%   \framesubtitle{Problem statement}
%
%   \pause
%
%   \begin{itemize}
%     \item We're trying to answer questions that require consulting
%       heterogeneous data sources.
%
%     \pause
%
%     \begin{itemize}
%       \item Literature with inconsistent, semi-structured data.
%
%       \pause
%
%       \item No standard naming convention.
%
%       \pause
%
%       \item No information about the reliability of the data sources.
%
%       \pause
%
%       \item Each data source has its specific browsing/querying mechanism (no
%         common interface.)
%     \end{itemize}
%   \end{itemize}
% \end{frame}
%
% \begin{frame}
%   \frametitle{Motivation}
%   \framesubtitle{Sample problem domain: \textbf{biorefinery}}
%
%   \begin{itemize}
%     \item Ligno-cellulosic biomass pre-treatment before enzymatic hydrolysis is
%       an essential step to obtain good yields.
%
%     \pause
%
%     \item Several pre-treatment principles available, but \textbf{no clear
%       criteria on how to choose the best one} taking into account environmental
%       sustainability for a given biomass and biorefinery product (e.g.
%       glucose.)
%   \end{itemize}
% \end{frame}
%
% \begin{frame}
%   \frametitle{Proposed solution}
%
%   \begin{itemize}
%     \item Represent scientific knowledge with ontologies using recommended
%       standardized tools and languages for such purposes (semantic web
%       technologies, RDF(S), OWL, etc.)
%
%     \pause
%
%     \item Develop an ontology and data management web application (e.g. the
%       \textbf{@Web platform}) that makes it easy for scientists to introduce
%       data from scientific publications into an ontology, execute queries
%       against an ontology, etc.
%
%     \pause
%
%     \item Create integrity constraints to automatically detect inconsistencies
%       and errors in scientific publications and to automatically classify
%       publications according to their topics.
%
%     \pause
%
%     \begin{itemize}
%       \item \textit{The focus of my internship!}
%     \end{itemize}
%
%   \end{itemize}
% \end{frame}
%
% \begin{frame}
%   \frametitle{An example of a termino-ontological resource}
%   \framesubtitle{Taken from the biorefinery application}
%
%   \begin{center}
%     \includegraphics[width=10cm]{termino-ontological-resource.jpg}
%   \end{center}
% \end{frame}
%
% \begin{frame}
%   \frametitle{Design goals for the core ontology}
%
%   \begin{itemize}
%     \item \textbf{Simple} so as to make the annotator's task easier.
%
%     \pause
%
%     \item \textbf{Generic} enough so that the approach can be applied to
%       different, unrelated domains.
%
%     \pause
%
%     \begin{itemize}
%       \item Proven in the domains of biorefinery and packaging selection.
%     \end{itemize}
%   \end{itemize}
% \end{frame}
%
% \begin{frame}
%   \frametitle{A sample relation}
%   \framesubtitle{Also from the biorefinery domain}
%
%   \begin{center}
%     \includegraphics[width=10cm]{relation.jpg}
%   \end{center}
% \end{frame}
%
% \begin{frame}
%   \frametitle{The \textbf{@Web} platform}
%   \framesubtitle{Exploring an ontology}
%
%   \begin{center}
%     \includegraphics[width=8cm]{atweb-ontology.jpg}
%   \end{center}
% \end{frame}
%
% \begin{frame}
%   \frametitle{The \textbf{@Web} platform}
%   \framesubtitle{Browsing documents}
%
%   \begin{center}
%     \includegraphics[width=10cm]{atweb-document.jpg}
%   \end{center}
% \end{frame}
%
% \begin{frame}
%   \frametitle{The \textbf{@Web} platform}
%   \framesubtitle{Querying an ontology: defining the search scope}
%
%   \begin{center}
%     \includegraphics[width=10cm]{atweb-query-1.jpg}
%   \end{center}
% \end{frame}
%
% \begin{frame}
%   \frametitle{The \textbf{@Web} platform}
%   \framesubtitle{Querying an ontology: search parameters}
%
%   \begin{center}
%     \includegraphics[width=10cm]{atweb-query-2.jpg}
%   \end{center}
% \end{frame}
%
% \begin{frame}
%   \frametitle{The \textbf{@Web} platform}
%   \framesubtitle{Querying an ontology: executing a query}
%
%   \begin{center}
%     \includegraphics[width=10cm]{atweb-query-3.jpg}
%   \end{center}
% \end{frame}
%
% \begin{frame}
%   \frametitle{The \textbf{@Web} platform}
%   \framesubtitle{Querying an ontology: results}
%
%   \begin{center}
%     \includegraphics[width=10cm]{atweb-query-4.jpg}
%   \end{center}
% \end{frame}
%
% \begin{frame}
%   \frametitle{The annotator's task}
%
%   \begin{itemize}
%     \item Given a scientific publication and a desired ontology, capture data
%       from the publication using the appropriate concepts in the ontology.
%
%     \pause
%
%     \item Create and update concepts in the ontology as they're discovered
%       during the annotation process (i.e. in an iterative fashion.)
%
%     \pause
%
%     \item Write and edit \textbf{guidelines} associated to each concept
%       explaining when and how a concept should be used.
%   \end{itemize}
% \end{frame}
%
% \begin{frame}
%   \frametitle{An example of data captured from a scientific publication}
%
%   \begin{center}
%     \includegraphics[width=11cm]{table.jpg}
%   \end{center}
% \end{frame}
%
% \begin{frame}
%   \frametitle{A sample guideline}
%
%   \begin{center}
%     \includegraphics[width=10cm]{guideline.jpg}
%   \end{center}
% \end{frame}
%
% \begin{frame}
%   \frametitle{Some sample guidelines that can be easily translated into SPARQL
%   constraints}
%   \framesubtitle{Integrity constraints}
%
%   \begin{itemize}
%     \item \textit{``The output quantity of a step is equal to the sum of the
%       quantity of water used and the quantity of biomass present in the
%     step.''}
%
%     \pause
%
%     \item \textit{``The second milling step must give an “Output solid
%       constituent size” smaller than 0,5-1 mm.''}
%   \end{itemize}
% \end{frame}
%
% \begin{frame}
%   \frametitle{Some sample guidelines that can be easily translated into SPARQL
%   constraints}
%   \framesubtitle{Classification constraints}
%
%   \begin{itemize}
%     \item \textit{``Topic Bioref-PM-PC-UFM-PS : included experiments are
%       composed of a pre-milling step, followed by a physico-chemical treatment,
%     then by an ultrafine milling step (ball milling, wet disk milling, etc.), a
%   press and separation step (washing and filtration), and finally the enzymatic
% hydrolysis step. This topic requires a press and separation step because there
% are a lot of effluents in the physico-chemical step or because the milling is
% made with effluent. The second milling step must give an “Output solid
% constituent size” smaller than 0,5-1 mm. (en)''}
%   \end{itemize}
% \end{frame}
%
% \begin{frame}
%   \frametitle{Examples of guidelines that \textbf{cannot} be easily
%   translated into SPARQL constraints}
%
%   \begin{itemize}
%     \item \textit{``In all treatments, when the authors indicate ``overnight'',
%       we considered a duration treatment between 10 and 15 hours''}
%
%     \pause
%
%     \item \textit{``Furthermore, we consider that the glucose rate equals to
%       glucan rate divided by 0.9.''}
%   \end{itemize}
% \end{frame}
%
% \begin{frame}
%   \frametitle{Statistics}
%   \framesubtitle{A promising approach}
%
%   In the biorefinery ontology alone we have:
%
%   \begin{itemize}
%     \item 11 occurrences of the phrase \textit{``equal to''}
%     \item 5 occurrences of the phrase \textit{``equals to''}
%     \item 11 occurrences of the phrase \textit{``sum of''}
%     \item 3 occurrences of the phrase \textit{``divided by''}
%     \item 2 occurrences of the phrase \textit{``multiplied by''}
%   \end{itemize}
%
%   spread across guidelines associated with 30 relation concepts.
%
%   \vspace{1em}
%
%   \textbf{At least 10 of them can be easily translated into SPARQL constraints.}
% \end{frame}

\begin{frame}
  \begin{center}
    \Huge{Thanks!}
  \end{center}
\end{frame}

\end{document}
